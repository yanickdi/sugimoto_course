\documentclass{beamer}

\mode<presentation>
{
  \usetheme{CambridgeUS}      % or try Darmstadt, Madrid, ...
  \usecolortheme{default} % or try albatross, beaver, crane, ...
  \usefonttheme{default}  % or try serif, structurebold, ...
  \setbeamertemplate{navigation symbols}{}
  \setbeamertemplate{caption}[numbered]
} 

\usepackage[english]{babel}
\usepackage[utf8x]{inputenc}

\title[BSP06 - Manipulierter Würfel]{Manipulierter Würfel}
\author{Dickbauer Y., Moser P., Perner M.}
\institute{PS Computergestützte Modellierung, WS 2016/17}
%\date{Date of Presentation}

\begin{document}

\begin{frame}
  \titlepage
\end{frame}

% Uncomment these lines for an automatically generated outline.
\begin{frame}{Outline}
  \tableofcontents
\end{frame}

\section{Aufgabenstellung}
\begin{frame}{Aufgabenstellung}
Ein manipulierter Würfel soll geworfen werden. 1000x würfeln. Es gelten folgende Wahrscheinlichkeiten
\begin{center}
\begin{tabular}{c|c|c|c|c|c|c}
Augenzahl & 1 & 2 & 3 & 4 & 5 & 6 \\ 
\hline 
Wahrscheinlichkeit & 1/10 & 1/20 & 1/5 & 1/10 & 1/2 & 1/20 \\ 
\end{tabular} 
\end{center}
\begin{itemize}
  \item Eingabe: -
  \item Output: Wie oft kommt es vor, dass 3x hintereinander die Zahl "3" gewürfelt wird (3333 = zweimal 3333).
\end{itemize}

\end{frame}

\section{Flow Chart}
\begin{frame}{Flow Chart}
	\centering
  	\includegraphics[scale=0.27]{BSP06_Flow_Chart_1.pdf}
\end{frame}
\section{Flow Chart}

\section{Programmcode}
\subsection{Main Funktion}
\begin{frame}[fragile]{Main Funktion - Programmeinstieg}
  \begin{lstlisting}[language=python]
RIGGED_DICE_PROBS = (1/10, 1/20, 1/5, 1/10, 1/2, 1/20)
NUMBER_OF_THROWS = 1000

def main():
    count = 0 #result
    subsequent = 0 #how often did we see it at the actual position
    for i in range(NUMBER_OF_THROWS):
        rigged_dice = loaded_random_choice(RIGGED_DICE_PROBS) + 1
        if rigged_dice == CHECK_DICE:
            # we've got one more
            subsequent += 1
            if subsequent >= AMOUNT_OF_TANDEMS:
                count += 1
        else:
            # that's the wrong dice -> set actual amount of subsequents back to zero
            subsequent = 0
            
    print('Anzahl an {} mal hintereinander eine {}: {}'.format(
        AMOUNT_OF_TANDEMS, CHECK_DICE, count))
\end{lstlisting}
\logopythonbottom
\end{frame}

\subsection{Verwendete Funktionen}
\begin{frame}[fragile]{Funktion loaded\_random\_choice(..)}
  \begin{itemize}
    \item Diese Funktion verlangt eine WSKL Liste als Eingabeparameter
    \item Gibt einen Index zurück, welcher 0 bis $\left\vert{probality\_list}\right\vert-1$ sein kann.
    \item Diese Indizes haben eine gewichtete WSKL, welche jeweils an der Position in der Eingabeliste steht
    \item Beispiel probility\_list := [ 0.9, 0.1 ]  $\Rightarrow$ mit p=90\% wird 0 zurückgegeben, p=10\% für 1
  \end{itemize}
  \begin{lstlisting}[language=python]
def loaded_random_choice(probability_list):
    n = len(probability_list)
    random_number = random.random()
    cum_p = 0
    for i in range(n):
        cum_p += probability_list[i]
        if cum_p > random_number:
            return i
    return None
\end{lstlisting}
\logopythonbottom
\end{frame}	
%\begin{frame}[fragile]{Funktion random\_number\_from\_interval(..)}
  \begin{itemize}
    \item Diese Funktion verlangt zwei Eingabeparameter \textit{lower} und \textit{upper}
    \item Gibt eine (pseudo)Zufallszahl (\textit{float}) im Intervall  [\textit{lower}, \textit{upper}) zurück 
    \item \textit{random.random()} ist eine Funktion der Python Standardbibliothek, welche ein Zufallszahl (\textit{float}) im Intervall [\textit{lower}, \textit{upper}) zurück gibt
    \item Mersenne Twister Methode wird als Generator der ZZ verwendet\footnote[frame] {\scriptsize\url{https://docs.python.org/3.5/library/random.html}} \footnote[frame] {\scriptsize\url{https://en.wikipedia.org/wiki/Mersenne_Twister}}
  \end{itemize}
  \begin{lstlisting}[language=python]
def random_number_from_interval(lower, upper):
    val = random.random()
    return lower + (upper -lower) * val
\end{lstlisting}
\logopythonbottom
\end{frame}	
%\begin{frame}[fragile]{Funktion user\_input(input\_vars, [use\_defaults])}
  \begin{itemize}
  	\item Diese Funktion verlang vom User die geforderten Eingabeparameter und gibt diese als von der Programmiererin gewünschten Datentyp wieder zurück
    \item Funktion verlangt als ersten Eingabeparameter die Liste \textit{input\_vars}
    \item Falls \textit{use\_defaults == True} wird der User nicht nach Eingabe gefragt (Dient zum Testen)
    \item Diese Liste besteht wiederrum aus Listen mit je Länge = 3:
    \begin{itemize}
    	\item 0: Text, welcher dem User ausgegeben wird
    	\item 1: Datentyp (int/float/str)
    	\item 2: Default value: Dieser Wert wird zurueckgegeben, falls \textit{use\_defaults == True}
    \end{itemize}
  \end{itemize}
  \begin{lstlisting}[language=python]
x, y = user_input((
    ('Geben Sie einen X Wert ein', int, 10),
    ('Geben Sie einen Y Wert ein', int,  5), False):
  \end{lstlisting}
  \logopythonbottom
\end{frame}	

\section{Beispiel}
\begin{frame}[fragile]{Beispiel anhand fixer Zufallszahlen}
\begin{itemize}
\item Annahme der Zufallszahlen wie folgt:
\end{itemize}
\begin{center}
  \begin{tabular}{|c|c|c|c|c|c|c|c}
  \hline 
  iteration & 0 & 1 & 2 & 3 & 4 & 5-999 \\ 
  \hline 
  ZZ      & 0.05 & 0.21 & 0.20 & 0.22 & 0.09 & 0.09\\ 
  rigged\_dice & 1 & 3 & 3 & 3 & 1 & 1\\ 
  \hline 
  \end{tabular} 
\end{center}
\begin{easylist}
\ListProperties(Hide=100, Hang=true, Progressive=3ex, Style*= ,
Style2*=$\bullet$ ,Style3*=$\circ$ ,Style4*=\tiny$\blacksquare$ )
& i := 0
&& rigged\_dice != 3 $\Rightarrow$ subsequent = 0, count = 0
& i := 1
&& rigged\_dice == 3 $\Rightarrow$ subsequent = 1, count = 0
& i := 2
&& rigged\_dice == 3 $\Rightarrow$ subsequent = 2, count = 0
\end{easylist}
\end{frame}

\begin{frame}[fragile]{Beispiel anhand fixer Zufallszahlen}
\begin{easylist}
\ListProperties(Hide=100, Hang=true, Progressive=3ex, Style*= ,
Style2*=$\bullet$ ,Style3*=$\circ$ ,Style4*=\tiny$\blacksquare$ )
& i := 3
&& rigged\_dice == 3 $\Rightarrow$ subsequent = 3 $\Rightarrow$ count = 1
& i := 4
&& rigged\_dice != 3 $\Rightarrow$ subsequent = 0, count = 1
\end{easylist}
Nach 1000 Iteration ist count = 1, also genau 1x 333 hintereinander
\end{frame}

\begin{frame}[fragile]{Anhang: Modifikation des Source Codes um Demo Beispiel zu erhalten}
  \begin{lstlisting}[language=python]
  #  Fuege folgenden Code vor random_number_from_interval() in lib.py ein:
ZZ = [0.05, 0.21, 0.20, 0.22] + [0.01]*1000
i = -1
def my_rand():
    global i
    i += 1
    return ZZ[i]
random.random = my_rand
  \end{lstlisting}
\logopythonbottom
\end{frame}
\end{document}
