\documentclass{beamer}

\mode<presentation>
{
  \usetheme{CambridgeUS}      % or try Darmstadt, Madrid, ...
  \usecolortheme{default} % or try albatross, beaver, crane, ...
  \usefonttheme{default}  % or try serif, structurebold, ...
  \setbeamertemplate{navigation symbols}{}
  \setbeamertemplate{caption}[numbered]
} 

\usepackage[english]{babel}
\usepackage[utf8x]{inputenc}

\title[BSP25 - Zufallszahlenüberprüfung]{Zufallszahlenüberprüfung}
\author{Dickbauer Y., Moser P., Perner M.}
\institute{PS Computergestützte Modellierung, WS 2016/17}
%\date{Date of Presentation}

\begin{document}

\begin{frame}
  \titlepage
\end{frame}

\begin{frame}{Outline}
  \tableofcontents
\end{frame}

\section{Aufgabenstellung}
\begin{frame}{Aufgabenstellung}
Erzeugen Sie mit einem gemischten Kongruenzgenerator für verschiedene Parameter 100 Zufallszahlen zwischen 0 und 1. Die Parameter für den gemischten Kongruenzgenerator sollen hierbei flexibel eingegeben werden können oder automatisch in vorgegebenen Intervallen untersucht werden. Die resultierenden Zufallszahlen sollen mittels $\chi^2-Anpassungstest$ und Runtest auf Unabhängigkeit geprüft werden (siehe Vorlesungs-Folien). \\~\\
Die Berechnung der Häufigkeiten ($\chi^2-Anpassungstest$) und Run-Länge, der zugehörigen $\chi^2$-Testgröße sowie der Vergleich mit dem korrespondierenden Wert aus der $\chi^2-Tabelle$ (z.B. zum 95\% Signifikanzniveau) soll dabei automatisch erfolgen, wobei die Werte der $\chi^2-Tabelle$ hard-codiert werden können.
\end{frame}

\begin{frame}{Aufgabenstellung}

\begin{itemize}
  \item Eingabe: Parameter für gemischten Kongruenzgenerator
  \vspace{1cm}
  \item Output: Zufallszahlen, Anzahl an Werten je Bereich, Annahme oder Ablehnung gemäß Tests
\end{itemize}
\end{frame}

\section{Erklärung}
\begin{frame}{$\chi^2-Anpassungstest$ - gemischte Kongruenzmethode}
	\begin{enumerate}
		\item Input n … Number of Random Numbers generated
		\item Classification into $\sqrt{n}$ classes of uniform size
		\item Calculate the frequency of random number in each class
		\item Calculate test statistic as the sum over all classes $\frac{(n_i - np)^2}{np}$
		\item Compare test statistic with rejection value from $\chi^2$ Table
		\item If X0 (test statistic) $\leftarrow$ reject value 
		\item random numbers are significantly independent
	\end{enumerate}
\end{frame}

\begin{frame}{Beispiel}
	\begin{itemize}
		\item $n = 100$
		\item $r = \sqrt{100} = 10 classes$
		\item $u_i = 0.34 \Rightarrow$ is beeing counted to class 4 $(0.34 - 0.4)$
		\item test statistic with all classes
	\end{itemize}
\end{frame}

\begin{frame}{Runtest}
	\begin{enumerate}
		\item Input: n = number of Runs
		\item a run is defined as $u_i$ (random numbers) that follow the order of being greater than the $u_{i-1}$
		\begin{itemize}
			\item e.g. $u_1 = 0.15, u_2 = 0.3, u_3 = 0.25$
		\end{itemize}
		\item Run in this case is 2
		\item Create classes according to length of run (here create class 2)
		\item Count number of run occurrances in each class
		\item Calculate $\chi_0 = \frac{1}{len(classes)!} - \frac{1}{(len(classes)+1)!}$
		\item Calculate $n * \chi$
		\item for test statistic calculate sum over all classes (length of class – np)/np 
		\item Test it the same way as before
	\end{enumerate}
\end{frame}

\subsection{Verwendete Funktionen}
%\begin{frame}[fragile]{Funktion loaded\_random\_choice(..)}
  \begin{itemize}
    \item Diese Funktion verlangt eine WSKL Liste als Eingabeparameter
    \item Gibt einen Index zurück, welcher 0 bis $\left\vert{probality\_list}\right\vert-1$ sein kann.
    \item Diese Indizes haben eine gewichtete WSKL, welche jeweils an der Position in der Eingabeliste steht
    \item Beispiel probility\_list := [ 0.9, 0.1 ]  $\Rightarrow$ mit p=90\% wird 0 zurückgegeben, p=10\% für 1
  \end{itemize}
  \begin{lstlisting}[language=python]
def loaded_random_choice(probability_list):
    n = len(probability_list)
    random_number = random.random()
    cum_p = 0
    for i in range(n):
        cum_p += probability_list[i]
        if cum_p > random_number:
            return i
    return None
\end{lstlisting}
\logopythonbottom
\end{frame}	

\section{Grafische Darstellung}

\end{document}
