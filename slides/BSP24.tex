\documentclass{beamer}

\mode<presentation>
{
  \usetheme{CambridgeUS}      % or try Darmstadt, Madrid, ...
  \usecolortheme{default} % or try albatross, beaver, crane, ...
  \usefonttheme{default}  % or try serif, structurebold, ...
  \setbeamertemplate{navigation symbols}{}
  \setbeamertemplate{caption}[numbered]
} 

\usepackage[english]{babel}
\usepackage[utf8x]{inputenc}

\title[BSP24 - Waschanlage]{Waschanlage}
\author{Dickbauer Y., Moser P., Perner M.}
\institute{PS Computergestützte Modellierung, WS 2016/17}
%\date{Date of Presentation}

\begin{document}

\begin{frame}
  \titlepage
\end{frame}

\begin{frame}{Outline}
  \tableofcontents
\end{frame}

\section{Aufgabenstellung}
\begin{frame}{Aufgabenstellung}
Bei der Planung einer Autowaschanlage muß der Besitzer entscheiden, wieviel Platz für
die wartenden Autos vorgesehen werden soll. Es wird geschätzt, dass die Kunden zufällig
(Poisson-Inputprozeß) mit einer mittleren Rate von einem Kunden alle 4 Minuten ankommen.
Ist der Warteplatz voll besetzt, würden ankommende Kunden mit ihren Fahrzeugen
wieder weiterfahren. \\~\\ Die Waschdauer ist in etwa exponentialveteilt mit einem Erwartungswert
von 3 Minuten. Vergleichen Sie den Erwartungswert des Anteils der potentiellen
Kunden, die wegen der fehlenden Wartemöglichkeit verloren gehen, wenn
\end{frame}

\begin{frame}{Aufgabenstellung}
\begin{enumerate}[(a)]
	\item kein Warteplatz
	\item zwei Warteplätze
	\item vier Warteplätze vorgesehen werden.
\end{enumerate}

\begin{itemize}
  \item Eingabe: -
  \item Output: Belegung der Warteplätze, Neuankünfte und Abfahrten
\end{itemize}
\end{frame}

\section{Flow Chart}
\begin{frame}{Flow Chart}
	\centering
  	\includegraphics[scale=0.22]{BSP24_FlowChart.pdf}
\end{frame}

\section{Erklärung}
\begin{frame}[fragile]{Phase I: Check Machine}
	\begin{itemize}
		\item Wenn remaining\_time\_in\_machine == 0:
		\begin{itemize}
			\item Auto fertig $\Rightarrow$ departures += 1
		\end{itemize}
		\item Wenn queue $>$ 0:
		\begin{itemize}
			\item Auto kommt in Waschanlage
			\item remaining\_time\_in\_machine = create\_processing\_time()
		\end{itemize}
	\end{itemize}
  \begin{lstlisting}[language=python]
def create_processing_time():
    # the processing time is exponential distributed with an exp val of 3 min.
    p_time = int(round(random_exp(3) * SIM_FREQUENCY, 0))
    if p_time == 0:
        p_time = 1
        print('Warning: processing time would be zero, increase simulation frequency!')
    return p_time
  \end{lstlisting}
\end{frame}

\begin{frame}[fragile]{Phase II: Check Arrivals}
	\begin{itemize}
		\item if next\_arrival == 0 
		\begin{itemize}
			\item Wenn Maschine frei $\Rightarrow$ sende Auto direkt in Waschanlage
			\item Wenn nicht frei $\Rightarrow$ sende Auto in die Queue (wenn max capacity queue noch nicht erreicht)
		\end{itemize}
		\item next\_arrival = create\_next\_arrival()
	\end{itemize}
  \begin{lstlisting}[language=python]
def create_arrival_time():
    # the next customer arrives after an average time of 4 minutes
    arr_time = int(round(random_poisson(4) * SIM_FREQUENCY, 0))  # (poisson distributed)
    if arr_time == 0:
        arr_time = 1
        print('Warning: arr_time would be zero, increase simulation frequency!')
    return arr_time
  \end{lstlisting}
\end{frame}


\subsection{Verwendete Funktionen}
%\begin{frame}[fragile]{Funktion loaded\_random\_choice(..)}
  \begin{itemize}
    \item Diese Funktion verlangt eine WSKL Liste als Eingabeparameter
    \item Gibt einen Index zurück, welcher 0 bis $\left\vert{probality\_list}\right\vert-1$ sein kann.
    \item Diese Indizes haben eine gewichtete WSKL, welche jeweils an der Position in der Eingabeliste steht
    \item Beispiel probility\_list := [ 0.9, 0.1 ]  $\Rightarrow$ mit p=90\% wird 0 zurückgegeben, p=10\% für 1
  \end{itemize}
  \begin{lstlisting}[language=python]
def loaded_random_choice(probability_list):
    n = len(probability_list)
    random_number = random.random()
    cum_p = 0
    for i in range(n):
        cum_p += probability_list[i]
        if cum_p > random_number:
            return i
    return None
\end{lstlisting}
\logopythonbottom
\end{frame}	

\section{Grafische Darstellung}
\begin{frame}{Ergebnis unserer Simulation}
	\centering
  	\includegraphics[scale=0.6]{BSP24_Plot.pdf}
\end{frame}

\end{document}
