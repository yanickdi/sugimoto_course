\begin{frame}[fragile]{Funktion user\_input(input\_vars, [use\_defaults])}
  \begin{itemize}
  	\item Diese Funktion verlang vom User die geforderten Eingabeparameter und gibt diese als von der Programmiererin gewünschten Datentyp wieder zurück
    \item Funktion verlangt als ersten Eingabeparameter die Liste \textit{input\_vars}
    \item Falls \textit{use\_defaults == True} wird der User nicht nach Eingabe gefragt (Dient zum Testen)
    \item Diese Liste besteht wiederrum aus Listen mit je Länge = 3:
    \begin{itemize}
    	\item 0: Text, welcher dem User ausgegeben wird
    	\item 1: Datentyp (int/float/str)
    	\item 2: Default value: Dieser Wert wird zurueckgegeben, falls \textit{use\_defaults == True}
    \end{itemize}
  \end{itemize}
  \begin{lstlisting}[language=python]
x, y = user_input((
    ('Geben Sie einen X Wert ein', int, 10),
    ('Geben Sie einen Y Wert ein', int,  5), False):
  \end{lstlisting}
  \logopythonbottom
\end{frame}	