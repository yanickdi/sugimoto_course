\documentclass{beamer}

\mode<presentation>
{
  \usetheme{CambridgeUS}      % or try Darmstadt, Madrid, ...
  \usecolortheme{default} % or try albatross, beaver, crane, ...
  \usefonttheme{default}  % or try serif, structurebold, ...
  \setbeamertemplate{navigation symbols}{}
  \setbeamertemplate{caption}[numbered]
} 

\usepackage[english]{babel}
\usepackage[utf8x]{inputenc}

\title[BSP21 - Lagerhaltung]{Lagerhaltung}
\author{Dickbauer Y., Moser P., Perner M.}
\institute{PS Computergestützte Modellierung, WS 2016/17}
%\date{Date of Presentation}

\begin{document}

\begin{frame}
  \titlepage
\end{frame}

\begin{frame}{Outline}
  \tableofcontents
\end{frame}

\section{Aufgabenstellung}
\begin{frame}{Aufgabenstellung}
Ein Großhändler steht vor folgendem Lagerhaltungsproblem: Die Lagerung eines Produktes
kostet 2 Euro pro Tag, eine Bestellung beim Lieferanten verursacht Kosten von
20 Euro (unabhängig von der Bestellmenge), der Verdienstentgang für ein nachgefragtes,
aber nicht vorhandenes Produkt beträgt 15 Euro.\\~\\
Die tägliche Nachfragemenge sei
\begin{enumerate}
\item binomialverteilt mit n = 7 und p = 0.5
\item poissonverteilt mit $\lambda$ = 3.
\end{enumerate}
\end{frame}

\begin{frame}{Aufgabenstellung}
Die Bestellregel für den Disponenten soll lauten: Bestelle q Stück, wenn weniger als s
Stück auf Lager sind. Bestimmen Sie dazu durch Simulation kostenoptimale Werte q, s
(Variation von q und s zwischen 2 und 8 Stück mit q $\geq$ s).\\~\\

\begin{itemize}
  \item Eingabe: -
  \item Output: Lagerbestand zu Periodenstart, Einkauf, Nachfrage, Verdienstentgang und
Gesamtkosten je Periode
\end{itemize}
\end{frame}

\begin{frame}{Das Problem der Lagerhaltung}
\begin{itemize}
  \item Es entstehen Kosten für
  \begin{itemize}
  	\item Lagerhaltung = Holding Cost
  	\item Ordering Cost
  	\item Vertriebsentgang = Penalty Cost
  \end{itemize}
  \item Es wird eine Bestellmenge q festgelegt
  \item Es wird ein safety stock s festgelegt
  \item falls stock < s wird q bestellt
\end{itemize}
\end{frame}

\section{Flow Chart}
\begin{frame}{Flow Chart}
	\centering
  	\includegraphics[scale=0.15]{BSP21_Flow_Chart.pdf}
\end{frame}

\section{Programmcode}
\subsection{Main Funktion}
\begin{frame}[fragile]{Main Funktion - Programmeinstieg}
  \begin{lstlisting}[language=python]
def main():
    print('Simulate Option a) - Binom:')
    sum_costs_binom = simulation(OPTION_BINOM)
    
    for i in range(5): print()
    
    print('Simulate Option b) - Poisson:')
    sum_costs_poisson = simulation(OPTION_POISSON)
    
    print('Sum costs of all periods of Option a: {}'.format(sum_costs_binom))
    print('Sum costs of all periods of Option b: {}'.format(sum_costs_poisson))
\end{lstlisting}
\logopythonbottom

  \begin{itemize}
  	\item Funktion simulation(): siehe .py file
  \end{itemize}
\end{frame}

\subsection{Verwendete Funktionen}
% \begin{frame}[fragile]{Funktion euclidean\_distance(p1, p2)}
  \begin{itemize}
    \item Diese Funktion verlangt zwei Punkte (x1, y1) (x2, y2) als Eingabeparameter
    \item Gibt die eukliedsche Distanz zurück
  \end{itemize}
  \begin{lstlisting}[language=python]
def euclidean_distance(point_1, point_2):
    """
        Calculates the euclidean distance between two points
        
        point_1: a tuple of (x,y) values
        point_2: a tuple of (x,y) values
    """
    delta_x = point_2[0] - point_1[0]
    delta_y = point_2[1] - point_1[1]
    return (delta_x ** 2 + delta_y ** 2) ** 0.5
\end{lstlisting}
\logopythonbottom
\end{frame}	
%\begin{frame}[fragile]{Funktion random\_number\_from\_interval(..)}
  \begin{itemize}
    \item Diese Funktion verlangt zwei Eingabeparameter \textit{lower} und \textit{upper}
    \item Gibt eine (pseudo)Zufallszahl (\textit{float}) im Intervall  [\textit{lower}, \textit{upper}) zurück 
    \item \textit{random.random()} ist eine Funktion der Python Standardbibliothek, welche ein Zufallszahl (\textit{float}) im Intervall [\textit{lower}, \textit{upper}) zurück gibt
    \item Mersenne Twister Methode wird als Generator der ZZ verwendet\footnote[frame] {\scriptsize\url{https://docs.python.org/3.5/library/random.html}} \footnote[frame] {\scriptsize\url{https://en.wikipedia.org/wiki/Mersenne_Twister}}
  \end{itemize}
  \begin{lstlisting}[language=python]
def random_number_from_interval(lower, upper):
    val = random.random()
    return lower + (upper -lower) * val
\end{lstlisting}
\logopythonbottom
\end{frame}	
 \begin{frame}[fragile]{Funktion random\_binom(n, p)}
  \begin{columns}
  \column{.49\textwidth}
    \begin{itemize}
  	\item Erzeugt binomialverteilte Zufallszahl
  	\item $ z := |\left\{ i|u_i<p \right\}|$
  \end{itemize}
  \begin{lstlisting}[language=python]
def random_binom(n, p):
    z = 0
    # create n random numbers
    # between [0,1]:
    for i in range(n):
        rand = random.random()
        z += 1 if rand < p else 0
    return z
\end{lstlisting}
\logopythonbottom
  \column{.49\textwidth}
    	\begin{figure}[h!]
    	\includegraphics[scale=0.5]{lib_random_binom_wahrscheinlichkeitsverteilung.png}
  			\caption{Wahrscheinlichkeitsverteilung der Binomialverteilung \tiny{(Wikipedia)}}
		\end{figure}
  \end{columns}
\end{frame}	
\begin{frame}[fragile]{Funktion random\_poisson($\lambda$)}
  \begin{columns}
  \column{.59\textwidth}
    \begin{itemize}
  	\item Erzeugt poisson verteilte Zufallszahl
  	\item $\prod_{k}^{i=1}u_i \leq e^{-\lambda} < \prod_{k-1}^{i=1}u_i$
  	\item Sobald obige Bedingung zutrifft ist P($\lambda$) verteilte Zufallszahl k - 1
  \end{itemize}
  \begin{lstlisting}[language=python]
def random_poisson(lambd):
  k = 0
  u_list = []
  middle_value = exp(-lambd)
  while True:
    k += 1
    u_list.append(random.random())
    left_side = product(u_list)
    right_side = product(u_list[0:-1])
    if left_side <= middle_value < right_side:
      break
  return k - 1
\end{lstlisting}
\logopythonbottom
  \column{.39\textwidth}
    	\begin{figure}[h!]
    	\includegraphics[scale=0.2]{lib_random_poisson_wahrscheinlichkeitsverteilung.png}
  			\caption{Wahrscheinlichkeitsverteilung der Poisson-Verteilung \tiny{(Wikipedia)}}
		\end{figure}
  \end{columns}
\end{frame}	

\end{document}
