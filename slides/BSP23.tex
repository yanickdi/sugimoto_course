\documentclass{beamer}

\mode<presentation>
{
  \usetheme{CambridgeUS}      % or try Darmstadt, Madrid, ...
  \usecolortheme{default} % or try albatross, beaver, crane, ...
  \usefonttheme{default}  % or try serif, structurebold, ...
  \setbeamertemplate{navigation symbols}{}
  \setbeamertemplate{caption}[numbered]
} 

\usepackage[english]{babel}
\usepackage[utf8x]{inputenc}

\title[BSP23 - Fertigungsstraße]{Fertigungsstraße}
\author{Dickbauer Y., Moser P., Perner M.}
\institute{PS Computergestützte Modellierung, WS 2016/17}
%\date{Date of Presentation}

\begin{document}

\begin{frame}
  \titlepage
\end{frame}

\begin{frame}{Outline}
  \tableofcontents
\end{frame}

\section{Aufgabenstellung}
\begin{frame}{Aufgabenstellung}
Eine Fertigungsstraße besteht aus 3 Maschinen. Die Straße erhält die zu bearbeitenden
Teile aus einem Rohlager (mit unendlichem Vorrat); die bearbeitenden Teile werden in
ein nachgeschaltetes Fertigteillager (mit ebenfalls unendlicher Kapazität) geliefert.\\~\\
Die normale Bearbeitungszeit eines Teiles beträgt pro Maschine eine Minute; bei 15% der
Teile tritt jedoch eine Störung von vier Minuten auf, so dass die gesamte Bearbeitungszeit
eines Teiles dann 5 Minuten pro Maschine beträgt.
\end{frame}

\begin{frame}{Aufgabenstellung}
Untersuchen Sie durch Simulation über 180 Minuten, ob es zweckmässig ist, die drei
Maschinen zu entkoppeln und zwischen den Maschinen ein Pufferlager einzurichten. Vergleichen
Sie dazu beispielsweise den Ausstoß (gefertigte Stückzahl), die Stillstandszeiten
und den Auslastungsgrad der Maschine sowie die durchschnittliche Bearbeitungszeit eines
Teiles auf der Fertigungsstraße. Stellen Sie im Rahmen der Präsentation den Ablauf
des Programmes anhand von selbstgewählten Zufallszahlen beziehungsweise mit einem
Gantt-Diagramm die Abhängigkeiten der Maschinen vor.
\begin{itemize}
  \item Eingabe: -
  \item Output: Verlauf von Produktion (Startzeit, Bearbeitungszeit, Endzeit je Produkt
und Maschine), Warteschlangenlänge bei Bearbeitung und Inspektion, sowie die
oben angeführten Kennzahlen.
Kennzahlen.
\end{itemize}
\end{frame}

\section{Flow Chart}
\begin{frame}{Flow Chart}
	\centering
  	\includegraphics[scale=0.4]{FlowChartTodo.pdf}
\end{frame}

\section{Programmcode}
\subsection{Main Funktion}
\begin{frame}[fragile]{Main Funktion - Programmeinstieg}
  \begin{lstlisting}[language=python]
def main():
	pass
\end{lstlisting}
\logopythonbottom
\end{frame}

\subsection{Verwendete Funktionen}
% \begin{frame}[fragile]{Funktion euclidean\_distance(p1, p2)}
  \begin{itemize}
    \item Diese Funktion verlangt zwei Punkte (x1, y1) (x2, y2) als Eingabeparameter
    \item Gibt die eukliedsche Distanz zurück
  \end{itemize}
  \begin{lstlisting}[language=python]
def euclidean_distance(point_1, point_2):
    """
        Calculates the euclidean distance between two points
        
        point_1: a tuple of (x,y) values
        point_2: a tuple of (x,y) values
    """
    delta_x = point_2[0] - point_1[0]
    delta_y = point_2[1] - point_1[1]
    return (delta_x ** 2 + delta_y ** 2) ** 0.5
\end{lstlisting}
\logopythonbottom
\end{frame}	
%\begin{frame}[fragile]{Funktion random\_number\_from\_interval(..)}
  \begin{itemize}
    \item Diese Funktion verlangt zwei Eingabeparameter \textit{lower} und \textit{upper}
    \item Gibt eine (pseudo)Zufallszahl (\textit{float}) im Intervall  [\textit{lower}, \textit{upper}) zurück 
    \item \textit{random.random()} ist eine Funktion der Python Standardbibliothek, welche ein Zufallszahl (\textit{float}) im Intervall [\textit{lower}, \textit{upper}) zurück gibt
    \item Mersenne Twister Methode wird als Generator der ZZ verwendet\footnote[frame] {\scriptsize\url{https://docs.python.org/3.5/library/random.html}} \footnote[frame] {\scriptsize\url{https://en.wikipedia.org/wiki/Mersenne_Twister}}
  \end{itemize}
  \begin{lstlisting}[language=python]
def random_number_from_interval(lower, upper):
    val = random.random()
    return lower + (upper -lower) * val
\end{lstlisting}
\logopythonbottom
\end{frame}	
\begin{frame}[fragile]{Funktion user\_input(input\_vars, [use\_defaults])}
  \begin{itemize}
  	\item Diese Funktion verlang vom User die geforderten Eingabeparameter und gibt diese als von der Programmiererin gewünschten Datentyp wieder zurück
    \item Funktion verlangt als ersten Eingabeparameter die Liste \textit{input\_vars}
    \item Falls \textit{use\_defaults == True} wird der User nicht nach Eingabe gefragt (Dient zum Testen)
    \item Diese Liste besteht wiederrum aus Listen mit je Länge = 3:
    \begin{itemize}
    	\item 0: Text, welcher dem User ausgegeben wird
    	\item 1: Datentyp (int/float/str)
    	\item 2: Default value: Dieser Wert wird zurueckgegeben, falls \textit{use\_defaults == True}
    \end{itemize}
  \end{itemize}
  \begin{lstlisting}[language=python]
x, y = user_input((
    ('Geben Sie einen X Wert ein', int, 10),
    ('Geben Sie einen Y Wert ein', int,  5), False):
  \end{lstlisting}
  \logopythonbottom
\end{frame}	

\section{Beispiel}
\begin{frame}[fragile]{Beispiel anhand fixer Zufallszahlen}
\begin{itemize}
\item Annahme der Zufallszahlen wie folgt:
\end{itemize}
\begin{center}
  \begin{tabular}{c|c|c|c|c}
  \hline 
  iteration & 0 & 1 & 2 & 3\\ 
  \hline 
  ZZ      & 1 & 2 & 3 & 4 \\ 
  \end{tabular} 
\end{center}
\begin{easylist}
\ListProperties(Hide=100, Hang=true, Progressive=3ex, Style*= ,
Style2*=$\bullet$ ,Style3*=$\circ$ ,Style4*=\tiny$\blacksquare$ )
& blub
\end{easylist}
\end{frame}

\end{document}
