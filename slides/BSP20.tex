\documentclass{beamer}

\mode<presentation>
{
  \usetheme{CambridgeUS}      % or try Darmstadt, Madrid, ...
  \usecolortheme{default} % or try albatross, beaver, crane, ...
  \usefonttheme{default}  % or try serif, structurebold, ...
  \setbeamertemplate{navigation symbols}{}
  \setbeamertemplate{caption}[numbered]
} 

\usepackage[english]{babel}
\usepackage[utf8x]{inputenc}

\title[BSP20 - Instandhaltung]{Instandhaltung}
\author{Dickbauer Y., Moser P., Perner M.}
\institute{PS Computergestützte Modellierung, WS 2016/17}
%\date{Date of Presentation}

\begin{document}

\begin{frame}
  \titlepage
\end{frame}

\begin{frame}{Outline}
  \tableofcontents
\end{frame}

\section{Aufgabenstellung}
\begin{frame}{Aufgabenstellung}
Ein Unternehmen hat ein Instandhaltungsproblem mit einem bestimmten komplexen Ausstattungsgegenstand.
Diese Anlage enthält 4 identische Vakuumröhren, die die Ursache
für die Schwierigkeiten sind. Das Problem sieht so aus, dass die Röhren ziemlich häufig
ausfallen, wodurch die Anlage für den Austausch abgeschaltet werden muss.\\

Die momentane Praxis besteht darin, die Röhren nur dann auszutauschen, wenn sie ausfallen.
Es wurde jedoch der Vorschlag gemacht, alle vier Röhren auszutauschen, wenn
eine von ihnen ausfällt, um die Häufigkeit zu reduzieren, mit der die Anlage stillgelegt
werden muss. Das Ziel besteht darin, diese beiden Alternativen bezüglich der Kosten zu
vergleichen.
\end{frame}

\begin{frame}{Aufgabenstellung}
Die entsprechenden Daten sehen folgendermaßen aus: Für jede Röhre besitzt die Laufzeit
bis zum Ausfall annähernd eine Normalverteilung ($\mu$ = 1500 Stunden, $\sigma$ = 500). Die
Anlage muss eine Stunde stillgelegt werden, um eine Röhre auszutauschen, bzw. zwei
Stunden, um alle vier Röhren zu ersetzen. Die Gesamtkosten, die mit der Stilllegung der
Anlage und dem Austausch der Röhren verbunden sind, betragen 100 Euro pro Stunde
plus 20 Euro für jede neue Röhre.
\end{frame}

\begin{frame}{Aufgabenstellung}
Simulieren Sie den Ablauf der beiden alternativen Politiken für die Simulationsdauer
von 55000 Stunden (einschließlich einer Stabilisierungsphase von 5000 Stunden), wobei
mit vier neuen Röhren begonnen werden soll, und stellen Sie auf Basis der Kosten einen
Vergleich der beiden Alternativen an. Stellen Sie im Rahmen der Präsentation den Ablauf
des Programmes anhand von selbstgewählten Zufallszahlen vor.
\begin{itemize}
  \item Eingabe: -
  \item Output: Status der Röhren und Kosten je Periode, kumulierte Kosten je Periode.
\end{itemize}
\end{frame}

\section{Flow Chart}
\begin{frame}{Flow Chart - Hauptprogramm}
	\centering
  	\includegraphics[scale=0.4]{BSP20_Flow_Chart.pdf}
\end{frame}

\begin{frame}{Flow Chart - Variante Tasche alle Röhren}
	\centering
  	\includegraphics[scale=0.2]{BSP20_Flow_Chart_all.pdf}
\end{frame}

\begin{frame}{Flow Chart - Variante Tasche nur die kaputte Röhre}
	\centering
  	\includegraphics[scale=0.15]{BSP20_Flow_Chart_one.pdf}
\end{frame}

\subsection{Verwendete Funktionen}
% \begin{frame}[fragile]{Funktion euclidean\_distance(p1, p2)}
  \begin{itemize}
    \item Diese Funktion verlangt zwei Punkte (x1, y1) (x2, y2) als Eingabeparameter
    \item Gibt die eukliedsche Distanz zurück
  \end{itemize}
  \begin{lstlisting}[language=python]
def euclidean_distance(point_1, point_2):
    """
        Calculates the euclidean distance between two points
        
        point_1: a tuple of (x,y) values
        point_2: a tuple of (x,y) values
    """
    delta_x = point_2[0] - point_1[0]
    delta_y = point_2[1] - point_1[1]
    return (delta_x ** 2 + delta_y ** 2) ** 0.5
\end{lstlisting}
\logopythonbottom
\end{frame}	
%\begin{frame}[fragile]{Funktion random\_number\_from\_interval(..)}
  \begin{itemize}
    \item Diese Funktion verlangt zwei Eingabeparameter \textit{lower} und \textit{upper}
    \item Gibt eine (pseudo)Zufallszahl (\textit{float}) im Intervall  [\textit{lower}, \textit{upper}) zurück 
    \item \textit{random.random()} ist eine Funktion der Python Standardbibliothek, welche ein Zufallszahl (\textit{float}) im Intervall [\textit{lower}, \textit{upper}) zurück gibt
    \item Mersenne Twister Methode wird als Generator der ZZ verwendet\footnote[frame] {\scriptsize\url{https://docs.python.org/3.5/library/random.html}} \footnote[frame] {\scriptsize\url{https://en.wikipedia.org/wiki/Mersenne_Twister}}
  \end{itemize}
  \begin{lstlisting}[language=python]
def random_number_from_interval(lower, upper):
    val = random.random()
    return lower + (upper -lower) * val
\end{lstlisting}
\logopythonbottom
\end{frame}	
 \begin{frame}[fragile]{Funktion random\_std(..)}
  \begin{itemize}
    \item Diese Funktion verlangt zwei optionale Parameter: $\mu$ und $\sigma$ welche per default auf 0 und 1 gesetzt sind
    \item Gibt eine normalverteilte Zufallszahl zurück
    \item N(0,1) := $\sqrt{-2 ln(u_1)} sin(2 \pi u_2)$ mit $u_1, u_2 = ZZ(0,1)$
    \item Anschließend Transformation: $\sigma N(0,1) + \mu$
  \end{itemize}
  \begin{lstlisting}[language=python]
def random_std(mean=0, sigma=1):
    """Returns a normally distrubed random number"""
    u1, u2 = random.random(), random.random()
    zz = (-2 * log(u1))**(1/2) * sin(2 * pi * u2)
    return sigma * zz + mean
\end{lstlisting}
\logopythonbottom
\end{frame}	

\section{Beispiel}
\begin{frame}[fragile]{Beispiel anhand fixer Zufallszahlen - Tausche alle Röhren}
	\centering
  	\includegraphics[scale=.5]{BSP20_Zufallszahlen_1.png}
\end{frame}

\begin{frame}[fragile]{Beispiel anhand fixer Zufallszahlen - Tausche einzelne Röhre}
	\centering
  	\includegraphics[scale=.5]{BSP20_Zufallszahlen_2.png}
\end{frame}




\end{document}
