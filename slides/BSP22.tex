\documentclass{beamer}

\mode<presentation>
{
  \usetheme{CambridgeUS}      % or try Darmstadt, Madrid, ...
  \usecolortheme{default} % or try albatross, beaver, crane, ...
  \usefonttheme{default}  % or try serif, structurebold, ...
  \setbeamertemplate{navigation symbols}{}
  \setbeamertemplate{caption}[numbered]
} 

\usepackage[english]{babel}
\usepackage[utf8x]{inputenc}

\title[BSP22 - Fertigungssystem]{Fertigungssystem}
\author{Dickbauer Y., Moser P., Perner M.}
\institute{PS Computergestützte Modellierung, WS 2016/17}
%\date{Date of Presentation}

\begin{document}

\begin{frame}
  \titlepage
\end{frame}

\begin{frame}{Outline}
  \tableofcontents
\end{frame}

\section{Aufgabenstellung}
\begin{frame}{Aufgabenstellung}
In einem Fertigungssystem werden Aufträge auf einer Maschine bearbeitet. Es gibt zwei
Typen von Produkten: Typ 1 (Typ 2) benötigt auf der Maschine eine Bearbeitungszeit, die
stetig gleichverteilt zwischen 2 und 6 min (1.5 und 4.5 min) liegt. Die Wahrscheinlichkeit,
dass ein Produkt vom Typ 1 ist, ist 0.4. Die Produkte kommen exponentialverteilt mit
Erwartungswert von 4 in das System. Die Kapazität der Warteschlange ist mit 5 Stück
begrenzt; Produkte, die in das System kommen, während die Warteschlangenkapazität
ausgelastet ist, werden aus dem System eliminiert.
\end{frame}

\begin{frame}{Aufgabenstellung}
Anschließend an die Bearbeitungsphase kommt eine Inspektion. Die Zeit, die man braucht,
um diese durchzuführen, ist für Produkttyp 1 (Produkttyp 2) gleichverteilt zwischen 3
und 5 min (1 und 3 min). Es wird überprüft, ob ein Produkt defekt ist oder nicht. Die
Wahrscheinlichkeit, dass ein Produkt defekt ist und somit aussortiert wird, ist 0.1.
\begin{enumerate}[(a)]
\item Zählen Sie die Stücke, die entfernt werden, weil die Kapazität der Warteschlange vor
der Maschine zu gering war, und die Anzahl der defekten Stücke.
\item Bestimmen Sie die durchschnittliche Länge der Warteschlange vor der Inspektionsstation
und die Auslastung der Maschine und der Inspektionsstation.
\item Wie lange brauchen die Produkte durchschnittlich, um durch das System geschleust
zu werden?
\end{enumerate}
\end{frame}

\begin{frame}{Aufgabenstellung}
Beginnen Sie die Simulation mit einer Aufwärmphase von 8h. Danach sollen alle statistischen
Werte gelöscht werden. Die Zeit der tatsächlichen Simulation ist 800h. Stellen Sie
im Rahmen der Präsentation den Ablauf des Programmes anhand von selbstgewählten
Zufallszahlen vor.

\begin{itemize}
  \item Eingabe: -
  \item Output: Verlauf von Produktion (Startzeit, Bearbeitungszeit, Endzeit je Produkt),
Warteschlangenlänge bei Bearbeitung und Inspektion, sowie die oben angeführten
Kennzahlen.
\end{itemize}
\end{frame}

\section{Flow Chart}
\begin{frame}{Flow Chart}
	\centering
  	\includegraphics[scale=0.25]{BSP22_Flow_Chart.pdf}
\end{frame}

\section{Programmcode}
\subsection{Main Funktion}
\begin{frame}[fragile]{Main Funktion - Programmeinstieg}
  \begin{lstlisting}[language=python]
def main():
    print('Starting the warming phase (8 hours):\n')
    simulate_system(8) 
    
    for i in range(5): print()
    print('Starting the real simulation (800h):')
    simulate_system(800)
\end{lstlisting}
\logopythonbottom

  \begin{itemize}
  	\item Funktion simulate\_system(): siehe .py file bzw. Flow Chart
  \end{itemize}
\end{frame}

\subsection{Verwendete Funktionen}
% \begin{frame}[fragile]{Funktion euclidean\_distance(p1, p2)}
  \begin{itemize}
    \item Diese Funktion verlangt zwei Punkte (x1, y1) (x2, y2) als Eingabeparameter
    \item Gibt die eukliedsche Distanz zurück
  \end{itemize}
  \begin{lstlisting}[language=python]
def euclidean_distance(point_1, point_2):
    """
        Calculates the euclidean distance between two points
        
        point_1: a tuple of (x,y) values
        point_2: a tuple of (x,y) values
    """
    delta_x = point_2[0] - point_1[0]
    delta_y = point_2[1] - point_1[1]
    return (delta_x ** 2 + delta_y ** 2) ** 0.5
\end{lstlisting}
\logopythonbottom
\end{frame}	
%\begin{frame}[fragile]{Funktion random\_number\_from\_interval(..)}
  \begin{itemize}
    \item Diese Funktion verlangt zwei Eingabeparameter \textit{lower} und \textit{upper}
    \item Gibt eine (pseudo)Zufallszahl (\textit{float}) im Intervall  [\textit{lower}, \textit{upper}) zurück 
    \item \textit{random.random()} ist eine Funktion der Python Standardbibliothek, welche ein Zufallszahl (\textit{float}) im Intervall [\textit{lower}, \textit{upper}) zurück gibt
    \item Mersenne Twister Methode wird als Generator der ZZ verwendet\footnote[frame] {\scriptsize\url{https://docs.python.org/3.5/library/random.html}} \footnote[frame] {\scriptsize\url{https://en.wikipedia.org/wiki/Mersenne_Twister}}
  \end{itemize}
  \begin{lstlisting}[language=python]
def random_number_from_interval(lower, upper):
    val = random.random()
    return lower + (upper -lower) * val
\end{lstlisting}
\logopythonbottom
\end{frame}	
\begin{frame}[fragile]{Funktion random\_number\_from\_interval(..)}
  \begin{itemize}
    \item Diese Funktion verlangt zwei Eingabeparameter \textit{lower} und \textit{upper}
    \item Gibt eine (pseudo)Zufallszahl (\textit{float}) im Intervall  [\textit{lower}, \textit{upper}) zurück 
    \item \textit{random.random()} ist eine Funktion der Python Standardbibliothek, welche ein Zufallszahl (\textit{float}) im Intervall [\textit{lower}, \textit{upper}) zurück gibt
    \item Mersenne Twister Methode wird als Generator der ZZ verwendet\footnote[frame] {\scriptsize\url{https://docs.python.org/3.5/library/random.html}} \footnote[frame] {\scriptsize\url{https://en.wikipedia.org/wiki/Mersenne_Twister}}
  \end{itemize}
  \begin{lstlisting}[language=python]
def random_number_from_interval(lower, upper):
    val = random.random()
    return lower + (upper -lower) * val
\end{lstlisting}
\logopythonbottom
\end{frame}	
\begin{frame}[fragile]{Funktion loaded\_random\_choice(..)}
  \begin{itemize}
    \item Diese Funktion verlangt eine WSKL Liste als Eingabeparameter
    \item Gibt einen Index zurück, welcher 0 bis $\left\vert{probality\_list}\right\vert-1$ sein kann.
    \item Diese Indizes haben eine gewichtete WSKL, welche jeweils an der Position in der Eingabeliste steht
    \item Beispiel probility\_list := [ 0.9, 0.1 ]  $\Rightarrow$ mit p=90\% wird 0 zurückgegeben, p=10\% für 1
  \end{itemize}
  \begin{lstlisting}[language=python]
def loaded_random_choice(probability_list):
    n = len(probability_list)
    random_number = random.random()
    cum_p = 0
    for i in range(n):
        cum_p += probability_list[i]
        if cum_p > random_number:
            return i
    return None
\end{lstlisting}
\logopythonbottom
\end{frame}	

 \begin{frame}[fragile]{Funktion random\_exp(n, p)}
  \begin{columns}
  \column{.49\textwidth}
    \begin{itemize}
  	\item Erzeugt exponential verteilte Zufallszahl
  	\item Gemäß Inversionsmethode:
  	\item $ZZ = -\frac{1}{\lambda} ln(u)$ mit u=R(0,1)
  \end{itemize}
  \begin{lstlisting}[language=python]
def random_exp(lambd):
    rand = random.random()
    return -(1/lambd) * log(rand)
\end{lstlisting}
\logopythonbottom
  \column{.49\textwidth}
    	\begin{figure}[h!]
    	\includegraphics[scale=0.3]{lib_random_exp_wahrscheinlichkeitsverteilung.png}
  			\caption{Wahrscheinlichkeitsverteilung der Exponentialverteilung \tiny{(Wikipedia)}}
		\end{figure}
  \end{columns}
\end{frame}	

\end{document}
