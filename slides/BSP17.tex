\documentclass{beamer}

\mode<presentation>
{
  \usetheme{CambridgeUS}      % or try Darmstadt, Madrid, ...
  \usecolortheme{default} % or try albatross, beaver, crane, ...
  \usefonttheme{default}  % or try serif, structurebold, ...
  \setbeamertemplate{navigation symbols}{}
  \setbeamertemplate{caption}[numbered]
} 

\usepackage[english]{babel}
\usepackage[utf8x]{inputenc}

\title[BSP17 - Warteschlangenmodell]{Warteschlangenmodell}
\author{Dickbauer Y., Moser P., Perner M.}
\institute{PS Computergestützte Modellierung, WS 2016/17}
%\date{Date of Presentation}

\begin{document}

\begin{frame}
  \titlepage
\end{frame}

\begin{frame}{Outline}
  \tableofcontents
\end{frame}

\section{Aufgabenstellung}
\begin{frame}{Aufgabenstellung}
Gegeben sei folgendes einfache Warteschlangenmodell: Für fixe Zeitintervalle gilt, dass
mit Wahrscheinlichkeit (Wk) p ein Kunde ankommt und mit Wk (1 − p) kein Kunde
ankommt. In jedem Intervall wird mit Wk (1 − q) ein Kunde fertig bedient und Wk q
nicht fertig bedient. Die Ankünfte und Bedienungen sind über die Perioden unabhängige
Ereignisse. Simulieren Sie das Warteschlangenmodell über N Perioden für verschiedene
Werte von p und q. Bestimmen Sie die durchschnittliche Länge der Warteschlange, die
mittlere Wartezeit und den Auslastungsgrad des Systems.

\begin{itemize}
  \item Eingabe: Anzahl an Simulationsdauer
  \item Output: Werte für Warteschlangenlänge zu Periodenstart, Veränderung während
der Periode und die oben angeführten Kennzahlen
\end{itemize}
\end{frame}

\end{document}
