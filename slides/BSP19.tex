\documentclass{beamer}

\mode<presentation>
{
  \usetheme{CambridgeUS}      % or try Darmstadt, Madrid, ...
  \usecolortheme{default} % or try albatross, beaver, crane, ...
  \usefonttheme{default}  % or try serif, structurebold, ...
  \setbeamertemplate{navigation symbols}{}
  \setbeamertemplate{caption}[numbered]
} 

\usepackage[english]{babel}
\usepackage[utf8x]{inputenc}

\title[BSP19 - Verderbliche Ware]{Verderbliche Ware}
\author{Dickbauer Y., Moser P., Perner M.}
\institute{PS Computergestützte Modellierung, WS 2016/17}
%\date{Date of Presentation}

\begin{document}

\begin{frame}
  \titlepage
\end{frame}

\begin{frame}{Outline}
  \tableofcontents
\end{frame}

\section{Aufgabenstellung}
\begin{frame}{Aufgabenstellung}
Die Produktionsmenge (in Stück) einer verderblichen Ware sei N(100000, 2000)-verteilt;
die nachgefragte Menge pro Woche sei N(110000, 20000)-verteilt. Der Gewinn je verkauftem
Stück betrage 3 Euro.
Da die erwartete Nachfrage größer als die erwartete Produktion ist, überlegt der Produzent,
eine Überstundenschicht einzulegen. Die produzierte Menge pro Woche der zusätzlichen
Schicht wird als N(10000, 200)-verteilt angenommen. Dabei entstehen zusätzliche
Produktionskosten von 80 Cent pro Stück. Jedes nichtverkaufte Stück verursacht Kosten
von 5 Euro. Durch Simulation über 50 Wochen soll entschieden werden, welche der
Alternativen günstiger ist.

\begin{itemize}
  \item Eingabe: -
  \item Output: je Periode die Gesamtnachfrage, Produktion ohne bzw. mit Überschicht,
Kosten und Gewinn, sowie Gesamtgewinn.
\end{itemize}
\end{frame}


\section{Flow Chart}
\begin{frame}{Flow Chart}
	\centering
  	\includegraphics[scale=0.25]{BSP19_Flow_Chart.pdf}
\end{frame}

\subsection{Verwendete Funktionen}
% \begin{frame}[fragile]{Funktion euclidean\_distance(p1, p2)}
  \begin{itemize}
    \item Diese Funktion verlangt zwei Punkte (x1, y1) (x2, y2) als Eingabeparameter
    \item Gibt die eukliedsche Distanz zurück
  \end{itemize}
  \begin{lstlisting}[language=python]
def euclidean_distance(point_1, point_2):
    """
        Calculates the euclidean distance between two points
        
        point_1: a tuple of (x,y) values
        point_2: a tuple of (x,y) values
    """
    delta_x = point_2[0] - point_1[0]
    delta_y = point_2[1] - point_1[1]
    return (delta_x ** 2 + delta_y ** 2) ** 0.5
\end{lstlisting}
\logopythonbottom
\end{frame}	
%\begin{frame}[fragile]{Funktion random\_number\_from\_interval(..)}
  \begin{itemize}
    \item Diese Funktion verlangt zwei Eingabeparameter \textit{lower} und \textit{upper}
    \item Gibt eine (pseudo)Zufallszahl (\textit{float}) im Intervall  [\textit{lower}, \textit{upper}) zurück 
    \item \textit{random.random()} ist eine Funktion der Python Standardbibliothek, welche ein Zufallszahl (\textit{float}) im Intervall [\textit{lower}, \textit{upper}) zurück gibt
    \item Mersenne Twister Methode wird als Generator der ZZ verwendet\footnote[frame] {\scriptsize\url{https://docs.python.org/3.5/library/random.html}} \footnote[frame] {\scriptsize\url{https://en.wikipedia.org/wiki/Mersenne_Twister}}
  \end{itemize}
  \begin{lstlisting}[language=python]
def random_number_from_interval(lower, upper):
    val = random.random()
    return lower + (upper -lower) * val
\end{lstlisting}
\logopythonbottom
\end{frame}	
%\begin{frame}[fragile]{Funktion user\_input(input\_vars, [use\_defaults])}
  \begin{itemize}
  	\item Diese Funktion verlang vom User die geforderten Eingabeparameter und gibt diese als von der Programmiererin gewünschten Datentyp wieder zurück
    \item Funktion verlangt als ersten Eingabeparameter die Liste \textit{input\_vars}
    \item Falls \textit{use\_defaults == True} wird der User nicht nach Eingabe gefragt (Dient zum Testen)
    \item Diese Liste besteht wiederrum aus Listen mit je Länge = 3:
    \begin{itemize}
    	\item 0: Text, welcher dem User ausgegeben wird
    	\item 1: Datentyp (int/float/str)
    	\item 2: Default value: Dieser Wert wird zurueckgegeben, falls \textit{use\_defaults == True}
    \end{itemize}
  \end{itemize}
  \begin{lstlisting}[language=python]
x, y = user_input((
    ('Geben Sie einen X Wert ein', int, 10),
    ('Geben Sie einen Y Wert ein', int,  5), False):
  \end{lstlisting}
  \logopythonbottom
\end{frame}	
 \begin{frame}[fragile]{Funktion random\_std(..)}
  \begin{itemize}
    \item Diese Funktion verlangt zwei optionale Parameter: $\mu$ und $\sigma$ welche per default auf 0 und 1 gesetzt sind
    \item Gibt eine normalverteilte Zufallszahl zurück
    \item N(0,1) := $\sqrt{-2 ln(u_1)} sin(2 \pi u_2)$ mit $u_1, u_2 = ZZ(0,1)$
    \item Anschließend Transformation: $\sigma N(0,1) + \mu$
  \end{itemize}
  \begin{lstlisting}[language=python]
def random_std(mean=0, sigma=1):
    """Returns a normally distrubed random number"""
    u1, u2 = random.random(), random.random()
    zz = (-2 * log(u1))**(1/2) * sin(2 * pi * u2)
    return sigma * zz + mean
\end{lstlisting}
\logopythonbottom
\end{frame}	

\end{document}
