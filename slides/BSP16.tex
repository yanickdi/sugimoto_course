\documentclass{beamer}

\mode<presentation>
{
  \usetheme{CambridgeUS}      % or try Darmstadt, Madrid, ...
  \usecolortheme{default} % or try albatross, beaver, crane, ...
  \usefonttheme{default}  % or try serif, structurebold, ...
  \setbeamertemplate{navigation symbols}{}
  \setbeamertemplate{caption}[numbered]
} 

\usepackage[english]{babel}
\usepackage[utf8x]{inputenc}

\title[BSP16 - Inversionsmethode]{Inversionsmethode}
\author{Dickbauer Y., Moser P., Perner M.}
\institute{PS Computergestützte Modellierung, WS 2016/17}
%\date{Date of Presentation}

\begin{document}

\begin{frame}
  \titlepage
\end{frame}

% Uncomment these lines for an automatically generated outline.
\begin{frame}{Outline}
  \tableofcontents
\end{frame}

\section{Aufgabenstellung}
\begin{frame}{Aufgabenstellung}
Erzeugen Sie mit Hilfe der Inversionsmethode N Zufallszahlen f¨ur die Zufallsverteilung
X, die durch ihre Verteilungsfunktion F(x) gegeben ist:
\begin{equation}
   f(x) =
   \begin{cases}
     0 & \text{für } x < 0 \\
     1 - \frac{1}{1+x^2} & \text{für } x \geq 0 \\
   \end{cases}
\end{equation}
\begin{itemize}
  \item Eingabe: Anzahl an Zufallszahlen
  \item Output: Zufallszahlen
\end{itemize}

\end{frame}

\begin{frame}{Grafische Darstellung der Funktion}
	\centering
  	\includegraphics[scale=0.5]{BSP16_plot_function.pdf}
\end{frame} 

\begin{frame}{Inversionsmethode}
\begin{equation}
   f(x) =
   \begin{cases}
     0 & \text{für } x < 0 \\
     1 - \frac{1}{1+x^2} & \text{für } x \geq 0 \\
   \end{cases}
\end{equation}
\begin{equation}
   f^{-1}(x) =
   \begin{cases}
     0 & \text{für } x = 0 \\
     \pm \sqrt{\frac{1}{1-x} - 1} & \text{für } 0 > x < 1 \\
   \end{cases}
\end{equation}
	\centering
  	\includegraphics[scale=0.3]{BSP16_plot_function_inverse.pdf}
\end{frame}

\section{Flow Chart}
\begin{frame}{Flow Chart}
	\centering
  	\includegraphics[scale=0.4]{BSP16_FlowChart.pdf}
\end{frame}

\end{document}
